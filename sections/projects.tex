\cvsection{Projetos}

\begin{cventries}
  \cventry
  {Projeto de Graduação em Telemática} % position
  {Impacto das Modulações do IEEE 802.15.4g na Qualidade de comunicação em ambiente de Smart Building} % Project
  {} % Empty location
  {2020} % date
  {
    \begin{cvitems} % Description(s) bullet points
      \item {Utilizando Sistemas Embarcados para verificar parâmetros de telecomunicações para analise de dados da qualidade do enlace sem fio.}
      \item {Tecnologias utilizadas: C/C++, Python 3, InfluxDB, GNU/Linux.}
      \item {Repositório: \url{https://github.com/GComPI-IFPB/openmote-fw}}
    \end{cvitems}
  }

  \cventry
  {Projeto pessoal} % position
  {MonsterAPI} % Project
  {} % Empty location
  {Em andamento} % date
  {
    \begin{cvitems} % Description(s) bullet points
      \item {Uma API RESTFul para consolidar conceitos de estudos de desenvolvimento Web, Golang e MongoDB}
      \item {Tecnologias utilizadas: Go Lang, MongoDB, API RESTFul.}
      \item {Repositório: \url{https://github.com/felipefbs/MonsterAPI}}
    \end{cvitems}
  }

  \cventry
  {Projeto pessoal} % position
  {Coronavírus BR} % Project
  {} % Empty location
  {2020} % date
  {
    \begin{cvitems} % Description(s) bullet points
      \item {Site desenvolvido em conjunto com colegas com o objetivo de informar o povo brasileiro sobre a pandemia da COVID-19.}
      \item {Tecnologias utilizadas: Typescript, ReactJS, Gatsbyjs, GraphQL.}
      \item {Repositório: \url{https://github.com/henry-ns/coronavirusbr}}
    \end{cvitems}
  }

  \cventry
  {Desafio} % position
  {Tim Maia Bot} % Project
  {} % Empty location
  {2020} % date
  {
    \begin{cvitems} % Description(s) bullet points
      \item {Bot músical para Discord que toca músicas do cantor Tim Maia desenvolvido para o Desafio333 da comunidade do Código Falado.}
      \item {Tecnologias utilizadas: Javascript, Discord.js.}
      \item {Repositório: \url{https://github.com/felipefbs/desafio333/tree/master/2020-Bot-Discord/felipefbs}}
    \end{cvitems}
  }

  \cventry
  {Desafio} % position
  {Tetris333} % Project
  {} % Empty location
  {2020} % date
  {
    \begin{cvitems} % Description(s) bullet points
      \item {O clássico jogo tetris desenvolvido em Typescript para o Desafio333 da comunidade do Código Falado.}
      \item {Tecnologias utilizadas: Typescript,  p5.js, ReactJS.}
      \item {Repositório: \url{https://github.com/henry-ns/tetris333}}
    \end{cvitems}
  }

  \cventry
  {Projeto Academico} % position
  {Manipulação de Braço Robótico Usando Acelerômetro e Microcontrolador ESP32 Aplicado ao Ensino da Robótica} % Project
  {} % Empty location
  {2019} % date
  {
    \begin{cvitems} % Description(s) bullet points
      \item {Utilização de um braço robótico controlado por microcontroladores para explicação de conceitos físicos como aceleração da gravidade aplicadas ao ensino básico.}
      \item {Tecnologias utilizadas: Arduino C.}
    \end{cvitems}
  }


\end{cventries}